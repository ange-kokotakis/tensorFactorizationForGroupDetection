\documentclass{article}
\usepackage{graphicx} % Required for inserting images
\usepackage[margin=1in]{geometry}
\usepackage{amsmath}
\usepackage{amsfonts}
\usepackage{listings} % to write code 
\usepackage{mathabx} % \boxslash


\title{Tensor Factorisation for Group Detection}
\author{ Ange Kokotakis, Romain Ramez \\ \small directed by Rodrigo Cabral Farias}
\date{November 2023}

\begin{document}

\maketitle

\begin{abstract}
    The aim of this project is to detect groups of people with a given dataset of interactions in a community.
    For this purpose we will represent the dataset as a tensor and use non-negative tensor factorization to approximate
    it as a product of different matrix from which we can get information about groups in this community.
\end{abstract}

\section{Introduction}

In this project, we use datasets similar to Figure \ref{dataset} in which each line represents an interaction between
two individuals: $id1$ and $id2$ at a certain $time$.

\begin{figure}[h]
    \centering
    \includegraphics[width=0.3\textwidth]{images/tableau_données.png}
    \caption{dataset}
    \label{dataset}
\end{figure}

From this dataset we create matrices $X \in \{0, 1\}^{I \times I}$ which represent interactions between people during a certain
interval of time where $I$ is equal to the number of people in the community and each coefficient is equal to :

\[
    X_{i,j} = 
    \begin{cases}
        1 & \text{if there is an interaction between person $i$ and $j$ during a given interval of time} \\
        0 & \text{otherwise}
    \end{cases}
\]

And then we stack these matrices with different interval $t_k$ of time to create our tensor $Y \in \{0, 1\}^{I \times I \times K}$
where K is the number of intervals of time. Finally the tensor containing the dataset has its coefficients equal to :

\[
    Y_{i,j,k} = 
    \begin{cases}
        1 & \text{if there is an interaction between person $i$ and $j$ at interval of time $t_k$} \\
        0 & \text{otherwise}
    \end{cases}
\]

\section{Model}

In order to approximate this tensor we create 3 matrices :
\[
    U \in \mathbb{R}_+^{I \times R}, \,
    V \in \mathbb{R}_+^{I \times R}, \,
    W \in \mathbb{R}_+^{K \times R}
\]

where :
\begin{itemize}
    \item[]
    \begin{itemize}
        \item $I$ is the number of individuals in the community,
        \item $K$ is the number of interval of time,
        \item $R$ is the number of groups in the community (we choose one randomly at first we will see later how to choose it correctly).
    \end{itemize}
\end{itemize}

The $U$ and $V$ matrices represent the membership level of a person to a certain group
and $W$ represents in which interval of time a group has been active.

% \[
%     L(U, V, W)=\|Y-[U, V, W]\|_{F}^2 +\lambda \|U-V\|_{F}^2
% \]

%\section{Multiplication Update Algorithm (MU)}

% \begin{lstlisting}[language=Python]
    % code
% \end{lstlisting}

We define $S \in \mathbb{R}_+^{I \times I \times K}$

\end{document}
%mettre en gras les matrices en gras voir mathbf